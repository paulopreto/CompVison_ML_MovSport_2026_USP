\documentclass[aspectratio=169]{beamer}

% Pacotes básicos e idioma
\usepackage[utf8]{inputenc}
\usepackage[portuguese]{babel}

% --- CHAMA O PACOTE DE ESTILO CUSTOMIZADO ---
\usepackage{preto}
\usepackage{fontawesome5}

% Caminho para as imagens (images/ e src/ para diagrama da rede)
\graphicspath{{./images/}{./src/}}

% --- Informações da Capa ---
\title{Visão Computacional e Deep Learning na Análise do Movimento de Atletas}
\subtitle{Perspectivas para a análise de desempenho}
\date{IV Simpósio de Tecnologia Aplicada à Análise de Desempenho Esportivo \\ 27 e 28 de Fevereiro de 2026}
\author{Prof. Dr. Paulo Roberto Pereia Santiago}
\institute{Escola de Educação Física e Esporte de Ribeirão Preto (EEFERP) - USP \\ Laboratório de Biomecânica e Controle Motor (LaBioCoM)}

\begin{document}

\begin{frame}
    \titlepage
\end{frame}

\begin{frame}{Agenda}
    \tableofcontents
\end{frame}

% ==========================================
% BLOCO 1: O Contexto, a Dor do Esporte e a Soberania Tecnológica
% ==========================================
\section{O Contexto e a Dor do Esporte}

\begin{frame}{Soberania Tecnológica: Criadores vs. Consumidores}
    \begin{itemize}
        \item O mercado esportivo brasileiro é historicamente dependente de tecnologias importadas (EUA, Europa, Austrália).
        \item Consumo de hardwares caríssimos e softwares fechados (\textit{Black Boxes}).
        \item O atual boom da Inteligência Artificial: ChatGPT, Gemini, Claude, Grok... onde está a tecnologia nacional?
        \item A urgência de deixarmos de ser apenas ``usuários \textit{premium}'' para nos tornarmos \textbf{desenvolvedores} das nossas próprias soluções.
    \end{itemize}
\end{frame}


% --- Slide 1: A Crítica ao Consumo Passivo ---
\begin{frame}{A Ilusão do Domínio: Somos Apertadores de Botões?}
    \begin{columns}[T]
        \begin{column}{0.65\textwidth}
            \begin{itemize}
                \item Temos a facilidade de comprar o \textit{software} mais caro e o \textit{hardware} mais moderno.
                \item \textbf{O Problema:} Sabemos onde clicar, mas não sabemos o que acontece no ``silício''.
                \item \textit{Analogia da Memória:} Tratamos a tecnologia como uma caixa preta de \textit{bits} consumíveis, não como geração de conhecimento.
            \end{itemize}
        \end{column}
        
        \begin{column}{0.35\textwidth}
            \centering
            \includegraphics[height=5.5cm, keepaspectratio]{pauloef_perdido.png}
        \end{column}
    \end{columns}
\end{frame}

% --- Slide 2: A Camada de Desenvolvimento (O Motor) ---
\begin{frame}{A Ilusão do Domínio: O que há por trás do botão?}
    \begin{center}
        \includegraphics[height=6.5cm, keepaspectratio]{paper1.png}
    \end{center}
\end{frame}

% --- Slide 3: O Nível do Detalhe (A Provocação) ---
\begin{frame}{A Ilusão do Domínio: A Ciência do Pixel}
    \begin{center}
        \includegraphics[height=4.5cm, keepaspectratio]{onepixel.png}
    \end{center}
    
    \vfill

    \begin{center}
        \begin{tcolorbox}[
            colback=red!5,
            colframe=red!75!black,
            title=Provocação Metodológica,
            width=0.95\textwidth,
            top=2pt, bottom=2pt
        ]
            \centering 
            \small \textbf{Como vamos discutir a complexidade da análise de desempenho se não sabemos como funciona o motor do algoritmo?} 
        \end{tcolorbox}
    \end{center}
\end{frame}

% --- Slide 4: A Matriz de Dados (A Realidade) ---
\begin{frame}{A Ilusão do Domínio: A Realidade dos Dados Brutos}
    \begin{center}
        \includegraphics[height=7cm, keepaspectratio]{images/terminal_xxd.png}
    \end{center}
\end{frame}

% --- Slide 5: O Hype / O Falso Domínio ---
\begin{frame}{Separando o \textit{Hype} da Ciência: O Falso Domínio}
    \begin{center}
        \includegraphics[height=4.2cm, keepaspectratio]{neobullshit.png}
        
        \vspace{0.2cm}
        \textbf{O \textit{Hype} Tecnológico} \\
        \small O perigo das ``caixas pretas'' e promessas mágicas sem validação e que você não é dono.
    \end{center}
\end{frame}

% --- Slide 6: A Ciência / O Desafio Real ---
\begin{frame}{Separando o \textit{Hype} da Ciência: O Pálido Ponto Azul}
    \begin{center}
        \includegraphics[height=4.2cm, keepaspectratio]{Pale_Blue_Dot.png}
        
        \vspace{0.2cm}
        \textbf{O Pálido Ponto Azul} \\
        \small O rigor metodológico: encontrar e rastrear o sinal (\textit{pixel}) verdadeiro no meio do ruído cósmico (ou da quadra).
    \end{center}
\end{frame}

\begin{frame}{A Nossa Resposta: Desenvolvimento e Validação Nacional}
    \begin{itemize}
        \item Rompendo a dependência de "caixas pretas" por meio da construção e validação de algoritmos de código aberto (\textit{open-source}).
        \item \textbf{Futebol de Campo:} Validação de métodos \textit{markerless} baseados em \textit{machine learning} contra a digitalização 3D tradicional para a análise cinemática do chute \cite{vieira2022automatic}.
        \item \textbf{Desportos de Combate:} Avaliação da precisão de arquiteturas como o \textit{OpenPose} para rastreamento de atletas em áreas oficiais de Taekwondo \cite{banks2024accuracy}.
        \item \textbf{Movimentos Complexos:} Desenvolvimento de \textit{pipelines} próprios (baseados em \textit{MediaPipe}) para quantificação da cinemática 2D em exercícios como o agachamento \cite{pereira2025markerless}.
    \end{itemize}
\end{frame}

% --- Slide Novo: Capa do Artigo Vieira et al. (2022) ---
\begin{frame}{Nossa Publicação: \textit{Markerless} no Futebol}
    \begin{center}
        \includegraphics[height=6.5cm, keepaspectratio]{vieira_capa.png}
    \end{center}
\end{frame}


% --- Slide Novo: Imagens do Artigo Vieira et al. (2022) ---
\begin{frame}{Validação em Cenário Real: O Chute no Futebol}
    \begin{itemize}
        \item Comparação visual do método \textit{markerless} automático contra a digitalização manual 3D tradicional \cite{vieira2022automatic}.
    \end{itemize}
    
    \vspace{0.3cm}
    
    \begin{columns}[c] % Alinhamento centralizado das colunas
        
        % Coluna Esquerda: Primeira Figura
        \begin{column}{0.5\textwidth}
            \begin{center}
                \includegraphics[width=0.95\textwidth, keepaspectratio]{vieira_imag1.png}
            \end{center}
        \end{column}
        
        % Coluna Direita: Segunda Figura
        \begin{column}{0.5\textwidth}
            \begin{center}
                \includegraphics[width=0.95\textwidth, keepaspectratio]{vieira_imag2.png}
            \end{center}
        \end{column}
        
    \end{columns}
\end{frame}


\begin{frame}{Código Aberto: Entendendo o Motor na Prática}
    \begin{itemize}
        \item Para romper a cultura da "caixa preta", todo o código discutido aqui é \textit{open-source}.
        \item Desenvolvemos um \textit{pipeline} completo em \textit{Python} (utilizando \textit{PyTorch}) para provar o conceito básico de \textit{Deep Learning}.
    \end{itemize}
    
    \vspace{0.3cm}
    
    \textbf{O que você encontrará no repositório (\texttt{src/}):}
    \begin{itemize}
        \item \textbf{Demonstração didática:} \texttt{img7x7.py} (matriz no terminal e bits); \\
        \texttt{png2raw.py} (PNG $\rightarrow$ RAW).
        \item \textbf{Geração de Dados:} \texttt{makedataset.py} (Construção de matrizes e \textit{datasets} sintéticos).
        \item \textbf{O Treinamento:} \texttt{trainblack7x7.py} (Uma Rede Neural Convolucional (CNN) feita do zero para prever coordenadas espaciais).
        \item \textbf{A Inferência:} \texttt{predictblackdot.py} (Ferramentas de terminal (CLI) para testar o rastreamento em novas imagens).
    \end{itemize}
    
    \vspace{0.4cm}
    
    \begin{center}
        \begin{tcolorbox}[
            colback=blue!5,
            colframe=blue!75!black,
            title={Acesse, Clone e Modifique:},
            boxsep=2pt, top=4pt, bottom=4pt, width=0.98\textwidth
        ]
            \centering
            \small \url{https://github.com/paulopreto/CompVison_ML_MovSport_2026_USP/tree/main/src}
        \end{tcolorbox}
    \end{center}
\end{frame}

\begin{frame}{O Paradigma Tradicional vs. Validade Ecológica}
    \begin{columns}
        \begin{column}{0.5\textwidth}
            \textbf{Padrão-Ouro (Laboratório)}
            \begin{itemize}
                \item Captura optoeletrônica com marcadores.
                \item Alta precisão, mas restrito a um ambiente rigidamente controlado.
                \item Invasivo e inviável em situação real de jogo.
            \end{itemize}
        \end{column}
        \begin{column}{0.5\textwidth}
            \textbf{Demanda Ecológica (Campo)}
            \begin{itemize}
                \item Avaliação durante o treino ou competição real.
                \item Necessidade de respostas rápidas para a equipe técnica.
                \item \textit{A "dor" do clube}: horas e horas de vídeo armazenadas, mas pouca capacidade de extrair métricas biomecânicas acionáveis.
            \end{itemize}
        \end{column}
    \end{columns}
\end{frame}


% ==========================================
% BLOCO 2: O Arsenal Tecnológico
% ==========================================
\section{O Arsenal Tecnológico: Visão Computacional e Deep Learning}

\begin{frame}{A Revolução da Captura sem Marcadores}
    \begin{itemize}
        \item Como as Redes Neurais Profundas viabilizaram a \textit{Markerless Motion Capture}.
        \item Democratização da extração de variáveis cinemáticas a partir de câmeras RGB comuns.
        \item Superando os ruídos e calibrando a predição para movimentos esportivos complexos.
    \end{itemize}
\end{frame}

\begin{frame}{Detecção e Rastreamento com YOLO}
    \begin{itemize}
        \item \textbf{YOLO (You Only Look Once):} Eficiência para identificar e rastrear múltiplos atletas em tempo real.
        \item O desafio da oclusão constante em desportos coletivos.
        \item Associações temporais e espaciais para manter a identidade do atleta ao longo do vídeo.
    \end{itemize}
    \begin{center}
        \includegraphics[height=4.0cm, keepaspectratio]{images/yolo_sayno.png}
    \end{center}

\end{frame}

\begin{frame}{Detecção e Rastreamento com YOLO}
    \begin{center}
        \includegraphics[height=5.0cm, keepaspectratio]{images/drone_comercial.png}
        
        \vspace{0.2cm}
        \small \textit{Faça o tracking aí então!}
    \end{center}
\end{frame}


\begin{frame}{Detecção e Rastreamento com YOLO}
    \begin{center}
        \includegraphics[height=5.0cm, keepaspectratio]{images/drone_i9.png}
        
        \vspace{0.2cm}
        \small \textit{E agora!}

    \end{center}
\end{frame}

% --- Slide: Estimativa de Pose com MediaPipe ---
\begin{frame}{Estimativa de Pose com MediaPipe}
    \begin{itemize}
        \item Modelagem do esqueleto 2D e 3D diretamente da imagem.
        \item Cálculo de ângulos articulares, velocidades e assimetrias na marcha/corrida.
        \item Da imagem bruta para o modelo biomecânico: validando os pontos anatômicos.
    \end{itemize}

    \begin{center}
        \includegraphics[height=5.0cm, keepaspectratio]{pk_binha2.png}
        
        \vspace{0.2cm}
        \small \textit{Extração automatizada de esqueleto e pontos anatômicos em tempo real.}
    \end{center}
\end{frame}


% --- Slide: As Limitações do "Out-of-the-Box" ---
\begin{frame}{Quando o Padrão Falha: Limites do MediaPipe}
    \begin{center}
        \includegraphics[height=4.5cm, keepaspectratio]{images/paper1_holanda.png}
        
        \vspace{0.3cm}
        \small \textit{\textcolor{red}{\faExclamationTriangle} \textbf{Falha de detecção:} Objeto/Alvo muito pequeno (Small) para os modelos pré-treinados padrão.}
    \end{center}
\end{frame}

% ------------------------------------------
% BLOCO: Deep Learning com PyTorch
% ------------------------------------------
\begin{frame}{Da Visão Computacional Clássica ao Deep Learning}
    \begin{itemize}
        \item Saímos da engenharia manual de \textit{features} (bordas, filtros, regras) para aprender diretamente a partir dos pixels.
        \item Redes neurais aprendem representações hierárquicas, tornando os modelos mais robustos a ruídos e variações do cenário real.
        \item O mesmo motor matemático serve da classificação simples à análise biomecânica avançada de movimentos.
    \end{itemize}
\end{frame}

\begin{frame}{Como uma CNN Enxerga}
    \begin{center}
        \includegraphics[height=4.0cm, keepaspectratio]{cnn.png}
    \end{center}
    
    \vspace{0.2cm}
    
    \begin{itemize}
        \item Convoluções + ReLU: extraem padrões locais (bordas, texturas, formas).
        \item \textit{Pooling}: reduz a dimensão e concentra a informação relevante.
        \item Camadas totalmente conectadas: transformam \textit{features} em decisão (classes, coordenadas, probabilidades).
    \end{itemize}
\end{frame}

% --- Slides: O que é a rede do repositório (visualização didática e diagrama) ---
\begin{frame}{O que é a rede do repositório? RastreadorDePontoCNN}
    \textbf{Objetivo didático:} uma CNN pequena que recebe uma imagem 7×7 em escala de cinza e prediz as coordenadas $(x, y)$ do ponto preto.
    \vspace{0.25cm}
    \begin{columns}[T]
        \begin{column}{0.5\textwidth}
            \textbf{Estrutura (fluxo dos dados):}
            \begin{itemize}
                \item \textbf{Entrada:} 1 canal, 7×7.
                \item \textbf{Bloco convolucional:} duas Conv2d (1→8, 8→16, kernel 3×3) + ReLU.
                \item \textbf{Bloco denso:} Flatten (784) → Linear 784→32 → ReLU → Linear 32→2.
                \item \textbf{Saída:} dois números (coordenadas $x$, $y$).
            \end{itemize}
        \end{column}
        \begin{column}{0.5\textwidth}
            \textbf{Como ``ver'' a rede na prática:}\\[0.2cm]
            O script \texttt{visualizar\_rede.py} do repositório:
            \begin{itemize}
                \item Imprime no terminal a estrutura em texto (camadas, formas dos tensores, total de parâmetros).
                \item Gera um diagrama em PNG do fluxo das camadas para uso em aula.
                \item Opcionalmente exporta o modelo para ONNX (visualização no Netron).
            \end{itemize}
        \end{column}
    \end{columns}
    \vspace{0.3cm}
    \centering
    \small \faTerminal\ \texttt{uv run visualizar\_rede.py -m modelo\_rastreador.pth}
\end{frame}

\begin{frame}{Visualização da rede: diagrama gerado pelo script}
    \begin{center}
        \includegraphics[width=\textwidth, keepaspectratio]{diagrama_rastreador_rede.png}
    \end{center}
    \vspace{0.2cm}
    \centering
    \small Diagrama produzido por \texttt{visualizar\_rede.py}; cada caixa é uma camada e as setas indicam o fluxo da informação (esquerda $\rightarrow$ direita).
\end{frame}

% --- Slide: PyTorch para Visão Computacional ---
\begin{frame}{PyTorch para Visão Computacional}
    \begin{itemize}
        \item Framework dominante em Visão Computacional e \textit{Machine Learning} pela flexibilidade do grafo dinâmico e facilidade de experimentação.
        \item Ecossistema robusto com \texttt{torchvision}: \textit{datasets}, modelos pré-treinados e \textit{transforms} prontos para uso.
        \item Forte integração com pesquisa e produção: do experimento no laboratório ao deploy em servidores e dispositivos embarcados.
    \end{itemize}

    \begin{center}
        \includegraphics[height=2.5cm, keepaspectratio]{images/pytorch.png}
    \end{center}
\end{frame}

\begin{frame}{Ecossistema de Visão no PyTorch}
    \begin{itemize}
        \item \textbf{Datasets}: acesso direto a ImageNet, CIFAR-10, MNIST, COCO e outros.
        \item \textbf{Modelos pré-treinados}: ResNet, EfficientNetV2, ConvNeXt, ViT, Swin, Mask R-CNN, entre outros.
        \item \textbf{Transforms}: pré-processamento e \textit{data augmentation} (resize, crop, rotação, normalização).
        \item \textbf{Ferramentas}: PyTorch Hub, tutoriais oficiais e exemplos de \textit{transfer learning}.
    \end{itemize}
\end{frame}

\begin{frame}{Workflow em PyTorch para Visão}
    \begin{enumerate}
        \item \textbf{Preparação de dados}: \texttt{Dataset} + \texttt{DataLoader} (imagens $\rightarrow$ tensores).
        \item \textbf{Definição do modelo}: CNNs ou Transformers com \texttt{torch.nn}.
        \item \textbf{Treinamento}: \texttt{autograd} + otimizadores (SGD/Adam) + loop de épocas.
        \item \textbf{Avaliação e inferência}: métricas em dados novos, salvamento e exportação (TorchScript/ONNX).
    \end{enumerate}
\end{frame}

\begin{frame}{Aplicações em Visão com Deep Learning}
    \begin{itemize}
        \item Classificação de imagens (ex.: tipo de gesto, tipo de jogada, padrões de movimento).
        \item Detecção de objetos (jogadores, bola, linhas, alvos, equipamentos).
        \item Segmentação de cenas (separar atleta/fundo, zonas do campo, regiões corporais).
        \item Modelos multimodais (visão + linguagem, modelos de difusão para simulação e geração de cenários).
    \end{itemize}
\end{frame}

\begin{frame}{Escolhendo o Modelo de Classificação}
    \begin{itemize}
        \item \textbf{CNNs modernas} (EfficientNetV2, ConvNeXt): equilíbrio entre acurácia e custo computacional, ideais para \textit{transfer learning}.
        \item \textbf{Transformers visuais} (ViT, Swin): brilham com pré-treinamento massivo, quando há muito dado e a prioridade é desempenho de ponta.
        \item \textbf{Modelos para borda/mobile} (MobileNetV3) vs. modelos mais pesados (ConvNeXt/ViT): \textit{trade-off} entre latência, memória e precisão.
    \end{itemize}
    
    \vspace{0.2cm}
    
    {\scriptsize Sempre considerar latência, consumo de memória e robustez, não apenas a acurácia top-1.}
\end{frame}

\begin{frame}{Começando na Prática}
    \begin{itemize}
        \item Tutorial oficial de \textit{Transfer Learning} em visão com PyTorch como ponto de partida.
        \item Uso de modelos pré-treinados de \texttt{torchvision.models} com poucas linhas de código.
        \item Adaptação das últimas camadas para o seu \textit{dataset} esportivo (fine-tuning parcial ou total).
    \end{itemize}
\end{frame}
% ==========================================
% BLOCO 3: Mão na Massa, Projetos Nacionais e Ferramentas Abertas
% ==========================================
\section{Mão na Massa: Validações Científicas do LaBioCoM}

% --- Slide: vailá Multimodal Toolbox ---
\begin{frame}[fragile]{\textit{vailá} Multimodal Toolbox: Integração e Liberdade \cite{santiago2024vaila}}
    \begin{columns}[c]
        \begin{column}{0.55\textwidth}
            \begin{itemize}
                \item Ecossistema \textit{open-source} em \textit{Python} que resolve a complexidade da integração de dados biomecânicos.
                \item Liberta o pesquisador e o analista das ``caixas pretas'' e do ``pesadelo'' de sincronizar diferentes fontes.
            \end{itemize}
            
            \vspace{0.3cm}
            \centering
            \begin{tcolorbox}[colback=blue!5, colframe=blue!75!black, boxsep=1pt, left=2pt, right=2pt, top=2pt, bottom=2pt]
                \centering \footnotesize \faGithub \hspace{1mm} \href{https://github.com/vaila-multimodaltoolbox/vaila}{github.com/vaila-multimodaltoolbox/vaila}
            \end{tcolorbox}
        \end{column}
        
        \begin{column}{0.45\textwidth}
            \begin{center}
                % Logo principal em destaque
                \includegraphics[width=0.7\textwidth, keepaspectratio]{vaila.png}
                                
                % Logos da stack técnica propositalmente pequenos
                \begin{center}
                    \includegraphics[height=0.9cm]{images/python.png} \hspace{0.4cm}
                    \includegraphics[height=0.9cm]{images/rust.png}
                \end{center}
            \end{center}
        \end{column}
    \end{columns}
\end{frame}

% ==========================================
% BLOCO 4: Perspectivas e Futuro
% ==========================================
\section{A Fronteira do Conhecimento e Perspectivas}
% --- Slide: A Provocação do VAR Semiautomático ---
\begin{frame}{A Provocação do VAR Semiautomático}
    \begin{itemize}
        \item \small A \textit{Premier League} adotou dezenas de \textit{smartphones} comuns para a Tecnologia de Impedimento Semiautomático (SAOT).
        \item \small \textbf{A Provocação:} Se a liga mais rica do mundo extrai métricas de elite usando celulares, o segredo não é o \textit{hardware} de 100 mil dólares, é o \textbf{algoritmo}.
        \item \small \textit{Se é possível fazer com um celular de prateleira, nós também podemos fazer aqui!}
    \end{itemize}

    \begin{columns}[c]
        \begin{column}{0.48\textwidth}
            \begin{center}
                \includegraphics[width=\textwidth, height=3.0cm, keepaspectratio]{images/variphone.png}
                \vspace{0.1cm} \\ \scriptsize \textit{Smartphones na borda do campo.}
            \end{center}
        \end{column}
        \begin{column}{0.48\textwidth}
            \begin{center}
                \includegraphics[width=\textwidth, height=3.0cm, keepaspectratio]{images/markerlessvariphone.png}
                \vspace{0.1cm} \\ \scriptsize \textit{Rastreamento markerless em tempo real.}
            \end{center}
        \end{column}
    \end{columns}

    \vspace{0.2cm}
    \centering
    \scriptsize
    \url{https://youtu.be/af6P0P25HQk} \quad \url{https://youtube.com/shorts/kxUme_SgOzc}

    \begin{center}
        \begin{tcolorbox}[colback=blue!5,colframe=blue!75!black,boxsep=1pt, top=2pt, bottom=2pt, width=0.98\textwidth]
            \centering \footnotesize O diferencial competitivo migrou da câmera para a inteligência em \textit{software}.
        \end{tcolorbox}
    \end{center}
\end{frame}

% --- Slide Novo: Relatividade e o Eclipse ---
\begin{frame}{Confirmação da Relatividade Geral: O Papel da História}
    \begin{columns}[T]
        \begin{column}{0.62\textwidth}
            \scriptsize
            A 1ª Guerra Mundial (1914-1918) impediu a primeira tentativa de validação da teoria de Einstein:
            \vspace{0.1cm}
            \begin{itemize}
                \item \textbf{1914 (Fracasso):} Erwin Freundlich liderou expedição à Crimeia. A eclosão da guerra causou a prisão da equipe pelos russos, abortando a missão.
                \item \textbf{A ``Sorte'' de Einstein:} Em 1914, os cálculos de Einstein ainda estavam incompletos. O fracasso evitou que dados imprecisos refutassem a teoria prematuramente.
                \item \textbf{1915:} Einstein finaliza as equações da Relatividade Geral.
                \item \textbf{1919 (Sucesso):} Arthur Eddington lidera expedições a \textbf{Sobral (Brasil)} e Ilha do Príncipe, confirmando o desvio da luz solar e validando Einstein mundialmente.
            \end{itemize}
        \end{column}
        
        \begin{column}{0.38\textwidth}
            \centering
            \includegraphics[height=2.0cm, keepaspectratio]{images/einsten_100anos.png}
            \vspace{0.2cm}
            
            \includegraphics[height=3.0cm, keepaspectratio]{images/einsten.png}
        \end{column}
    \end{columns}
\end{frame}

% --- Slide Final: Conclusão ---
\begin{frame}
    \vspace{1cm}
    \begin{center}
        \begin{tcolorbox}[
            colback=green!5,
            colframe=green!50!black,
            title=\centering \textit{Take-home Message},
            width=0.9\textwidth,
            boxsep=4pt, top=10pt, bottom=10pt
        ]
            \centering \Large \bfseries
            A IA e a Visão Computacional não substituem o treinador; elas fornecem a lente de precisão para a sua intuição e experiência.
        \end{tcolorbox}
    \end{center}
\end{frame}

% --- Slide Final: Agradecimentos ---
\begin{frame}
    \centering
    {\huge Obrigado!} \\
    
    \vspace{1.5em}
    
    \normalsize \textbf{Prof. Dr. Paulo Santiago} \\
    \small EEFERP-USP / LaBioCoM \\
    
    \vspace{1.5em}
    
    \begin{center}
        \begin{tcolorbox}[
            colback=blue!5, 
            colframe=blue!75!black, 
            width=0.8\textwidth,
            arc=2mm,
            auto outer arc
        ]
            \centering
            \small \faGithub \hspace{1mm} \href{https://github.com/paulopreto/CompVison_ML_MovSport_2026_USP}{https://github.com/paulopreto/CompVison\_ML\_MovSport\_2026\_USP}
        \end{tcolorbox}
    \end{center}
\end{frame}

% --- Slide de Referências ---
\begin{frame}[allowframebreaks]{Referências}
    \bibliographystyle{apalike} % Estilo da citação (ex: apalike, IEEEtran, etc)
    \bibliography{references}   % Chama o seu arquivo references.bib
\end{frame}

\end{document}