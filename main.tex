\documentclass[aspectratio=169]{beamer}

% Pacotes básicos e idioma
\usepackage[utf8]{inputenc}
\usepackage[portuguese]{babel}

% --- CHAMA O PACOTE DE ESTILO CUSTOMIZADO ---
\usepackage{preto}

% Caminho para as imagens 
\graphicspath{{./images/}}

% --- Informações da Capa ---
\title{Visão Computacional e Deep Learning na Análise do Movimento de Atletas}
\subtitle{Perspectivas para a análise de desempenho}
\date{IV Simpósio de Tecnologia Aplicada à Análise de Desempenho Esportivo \\ 27 e 28 de Fevereiro de 2026}
\author{Prof. Dr. Paulo Roberto Pereia Santiago}
\institute{Escola de Educação Física e Esporte de Ribeirão Preto (EEFERP) - USP \\ Laboratório de Biomecânica e Controle Motor (LaBioCoM)}

\begin{document}

\begin{frame}
    \titlepage
\end{frame}

\begin{frame}{Agenda}
    \tableofcontents
\end{frame}

% ==========================================
% BLOCO 1: O Contexto, a Dor do Esporte e a Soberania Tecnológica
% ==========================================
\section{O Contexto e a Dor do Esporte}

\begin{frame}{Soberania Tecnológica: Criadores vs. Consumidores}
    \begin{itemize}
        \item O mercado esportivo brasileiro é historicamente dependente de tecnologias importadas (EUA, Europa, Austrália).
        \item Consumo de hardwares caríssimos e softwares fechados (\textit{Black Boxes}).
        \item O atual boom da Inteligência Artificial: ChatGPT, Gemini, Claude, Grok... onde está a tecnologia nacional?
        \item A urgência de deixarmos de ser apenas ``usuários \textit{premium}'' para nos tornarmos \textbf{desenvolvedores} das nossas próprias soluções.
    \end{itemize}
\end{frame}

% --- Slide Novo 1: A Analogia do Motor ---
\begin{frame}{A Ilusão do Domínio: Somos Apertadores de Botões?}
    \begin{itemize}
        \item Temos a facilidade de comprar o \textit{software} mais caro e o \textit{hardware} mais moderno.
        \item \textbf{O Problema:} Sabemos onde clicar, mas não sabemos o que acontece no "silício".
        \item \textit{Analogia da Memória:} Tratamos a tecnologia como uma caixa preta de bits consumíveis, não como geração de conhecimento.
    \end{itemize}
    \begin{center}
        \begin{tcolorbox}[colback=red!5,colframe=red!75!black,title=Provocação Metodológica]
            \centering \large \textbf{Como vamos discutir a complexidade da análise de desempenho se não sabemos como funciona o motor do algoritmo?}
        \end{tcolorbox}
    \end{center}
\end{frame}

% --- Slide Novo 2: Exemplos de Protagonismo Externo ---
\begin{frame}{Enquanto isso, a vanguarda acontece lá fora...}
    \begin{itemize}
        \item \textbf{CVPR 2026 / Roboflow:} Desafios globais de visão computacional definindo os próximos padrões da indústria.
        \item \textbf{KAUST (Arábia Saudita):} Tornando-se o primeiro instituto de pesquisa da FIFA no Oriente Médio e Ásia.
        \item \textbf{\textit{Markerless Motion Capture}:} O estado da arte (\textit{state-of-the-art}) sendo ditado por laboratórios e empresas internacionais, como a Theia3D.
    \end{itemize}
    \vspace{0.1cm}
    \begin{center}
        \textbf{Por que não somos nós?} \\
        \vspace{0.1cm}
        \scriptsize 
        \url{https://blog.roboflow.com/roboflow20-vl-challenge-at-cvpr-2026/} \\
        \url{https://www.kaust.edu.sa/en/news/kaust-becomes-first-fifa-research-institute-in-the-middle-east-and-asia} \\
        \url{https://www.theiamarkerless.com/industries/sports-motion-capture} \\
        \url{https://www.linkedin.com/posts/theia-markerless_watch-markerless-motion-capture-for-applied-activity-7430277492789133312-dNnD}
    \end{center}
    \vspace{0.1cm}
    \begin{itemize}
        \item Precisamos decidir: coisas prontas ou desenvolver nossa própria inteligência?
    \end{itemize}
\end{frame}

\begin{frame}{De onde viemos: LaBioCoM e a Missão de Inovar}
    \begin{itemize}
        \item O Laboratório de Biomecânica e Controle Motor (LaBioCoM) na EEFERP-USP.
        \item Rompendo o ciclo de dependência: a universidade como polo de criação de tecnologia aplicada.
        \item Nossa missão: transpor o rigor metodológico do laboratório para a beira do campo.
        \item A sinergia fundamental: Academia + Clubes Esportivos + Empresas de Base Tecnológica.
    \end{itemize}
\end{frame}

\begin{frame}{O Paradigma Tradicional vs. Validade Ecológica}
    \begin{columns}
        \begin{column}{0.5\textwidth}
            \textbf{Padrão-Ouro (Laboratório)}
            \begin{itemize}
                \item Captura optoeletrônica com marcadores.
                \item Alta precisão, mas restrito a um ambiente rigidamente controlado.
                \item Invasivo e inviável em situação real de jogo.
            \end{itemize}
        \end{column}
        \begin{column}{0.5\textwidth}
            \textbf{Demanda Ecológica (Campo)}
            \begin{itemize}
                \item Avaliação durante o treino ou competição real.
                \item Necessidade de respostas rápidas para a equipe técnica.
                \item \textit{A "dor" do clube}: horas e horas de vídeo armazenadas, mas pouca capacidade de extrair métricas biomecânicas acionáveis.
            \end{itemize}
        \end{column}
    \end{columns}
\end{frame}

\begin{frame}{De onde viemos: LaBioCoM e a Pesquisa Aplicada}
    \begin{itemize}
        \item O Laboratório de Biomecânica e Controle Motor (LaBioCoM).
        \item Nossa missão: transpor o rigor do laboratório para a beira do campo.
        \item A aproximação entre a academia e as demandas reais do alto rendimento.
    \end{itemize}
\end{frame}

\begin{frame}{O Paradigma Tradicional vs. Validade Ecológica}
    \begin{columns}
        \begin{column}{0.5\textwidth}
            \textbf{Padrão-Ouro (Laboratório)}
            \begin{itemize}
                \item Captura com marcadores (ex: Vicon).
                \item Alta precisão, mas restrito ao ambiente controlado.
                \item Invasivo e inviável em situação de jogo.
            \end{itemize}
        \end{column}
        \begin{column}{0.5\textwidth}
            \textbf{Demanda Ecológica (Campo)}
            \begin{itemize}
                \item Avaliação durante o treino ou competição.
                \item Necessidade de respostas rápidas para a comissão técnica.
                \item \textit{A "dor" do clube}: ter horas de vídeo, mas poucos dados acionáveis.
            \end{itemize}
        \end{column}
    \end{columns}
\end{frame}


% ==========================================
% BLOCO 2: O Arsenal Tecnológico
% ==========================================
\section{O Arsenal Tecnológico: Visão Computacional e Deep Learning}

\begin{frame}{A Revolução da Captura sem Marcadores}
    \begin{itemize}
        \item Como as Redes Neurais Profundas viabilizaram a \textit{Markerless Motion Capture}.
        \item Democratização da extração de variáveis cinemáticas a partir de câmeras RGB comuns.
        \item Superando os ruídos e calibrando a predição para movimentos esportivos complexos.
    \end{itemize}
\end{frame}

\begin{frame}{Detecção e Rastreamento com YOLO}
    \begin{itemize}
        \item \textbf{YOLO (You Only Look Once):} Eficiência para identificar e rastrear múltiplos atletas em tempo real.
        \item O desafio da oclusão constante em desportos coletivos.
        \item Associações temporais e espaciais para manter a identidade do atleta ao longo do vídeo.
    \end{itemize}
    % Sugestão: Inserir um GIF/Vídeo curto do YOLO plotando as bounding boxes nos atletas
    \begin{center}
        \textit{[Inserir Vídeo/GIF do YOLO rastreando atletas aqui]}
    \end{center}
\end{frame}

\begin{frame}{Estimativa de Pose com MediaPipe}
    \begin{itemize}
        \item Modelagem do esqueleto 2D e 3D diretamente da imagem.
        \item Cálculo de ângulos articulares, velocidades e assimetrias na marcha/corrida.
        \item Da imagem bruta para o modelo biomecânico: validando os pontos anatômicos.
    \end{itemize}
    % Sugestão: Inserir um GIF do MediaPipe
    \begin{center}
        \textit{[Inserir Vídeo/GIF do MediaPipe extraindo o esqueleto de um atleta]}
    \end{center}
\end{frame}


% ==========================================
% BLOCO 3: Mão na Massa e Ferramentas Abertas
% ==========================================
\section{Mão na Massa: Projetos e Ferramentas Abertas}

\begin{frame}{O Desafio do Rastreamento no Futebol: Projeto UPro\_Soccer}
    \begin{itemize}
        \item Rastreando não apenas os atletas, mas o objeto de maior interesse: \textbf{a bola}.
        \item Desafios técnicos: bola pequena, alta velocidade, desfoque de movimento (\textit{motion blur}) e oclusão.
        \item Geração de trajetórias técnico-táticas combinadas para análise do jogo.
    \end{itemize}
    \begin{center}
        \textit{[Inserir Vídeo do tracking de bola do UPro\_Soccer]}
    \end{center}
\end{frame}

\begin{frame}{A Força do Código Aberto na Ciência do Esporte}
    \begin{itemize}
        \item A reprodutibilidade da pesquisa exige o compartilhamento das ferramentas.
        \item Fomentando uma comunidade colaborativa para acelerar a inovação.
        \item Como a cultura \textit{open-source} beneficia clubes com menores orçamentos.
    \end{itemize}
\end{frame}

\begin{frame}[fragile]{O \textit{vailá} Multimodal Toolbox}
    \begin{itemize}
        \item Desenvolvido para resolver o "pesadelo" da integração de dados cinemáticos e dinâmicos de diferentes fontes.
        \item Uma solução prática, em Python, para organizar pipelines de pesquisa biomecânica.
    \end{itemize}
    
\begin{lstlisting}[language=Python, caption={Exemplo simplificado de uso do vailá}]
import vaila as vl

# Carregando dados multimodais (Ex: cinemetria + plataforma de força)
dataset = vl.load_multimodal_data("coleta_atleta_01.csv")

# Sincronizando os sinais e aplicando filtro passa-baixa
dataset_sync = vl.synchronize_signals(dataset)
dataset_filtered = vl.apply_butterworth(dataset_sync, cutoff=10, fs=200)

# Calculando variáveis biomecânicas de interesse
metricas = vl.calculate_joint_angles(dataset_filtered)
\end{lstlisting}
\end{frame}


% ==========================================
% BLOCO 4: Perspectivas e Futuro
% ==========================================
\section{A Fronteira do Conhecimento e Perspectivas}

\begin{frame}{Fusão Multimodal e Dispositivos IoT}
    \begin{itemize}
        \item A Visão Computacional não trabalha sozinha.
        \item Sincronização de dados de vídeo com Wearables, sensores inerciais e dispositivos instrumentados (ex: blocos de partida IoT).
        \item O panorama completo do esforço e desempenho do atleta no cenário real.
    \end{itemize}
\end{frame}

\begin{frame}{Identificação de Talentos e Simulações Computacionais}
    \begin{itemize}
        \item Aplicação de \textit{machine learning} na identificação segura de talentos esportivos.
        \item O conceito de \textit{safe deselection}: cruzando evidências empíricas e simulações para evitar perdas de atletas com potencial de maturação tardia.
        \item A tecnologia reduzindo vieses humanos no processo de seleção.
    \end{itemize}
\end{frame}

\begin{frame}[standout]
    Take-home Message \\
    \vspace{0.5em}
    \Large A inteligência artificial e a visão computacional não substituem o treinador; elas fornecem a lente de precisão para a sua intuição e experiência.
\end{frame}

\begin{frame}[standout]
    Obrigado! \\
    \vspace{1em}
    \normalsize \textbf{Prof. Dr. Paulo Santiago} \\
    \normalsize EEFERP-USP / LaBioCoM \\
    \href{https://github.com/paulopreto/CompVison_ML_MovSport_2026_USP}{}
\end{frame}

\end{document}