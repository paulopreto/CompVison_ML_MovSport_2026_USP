\documentclass[aspectratio=169]{beamer}

% Pacotes básicos e idioma
\usepackage[utf8]{inputenc}
\usepackage[brazil]{babel}

% --- CHAMA O PACOTE CUSTOMIZADO ---
\usepackage{preto}

% Caminho para as imagens (garanta que logo_labiocom.png e GitHub-Logo.png estejam aqui)
\graphicspath{{./images/}}

% Título e autor
\title{Introdução ao Processamento de Imagens e Vídeos}
\author{\href{http://www.eeferp.usp.br/pessoas/docentes/paulo-roberto-pereira-santiago}{Prof. Paulo Roberto Pereira Santiago, Ph.D.}}
\date{\today}

\begin{document}

% Slide de título
\begin{frame}
    \titlepage
\end{frame}

% Slide: Introdução ao Processamento de Imagens
\begin{frame}{Introdução ao Processamento de Imagens}
    \begin{itemize}
        \item O processamento de imagens e vídeos envolve técnicas para manipular e analisar imagens digitais.
        \item Nesta aula, vamos explorar conceitos básicos usando exemplos em Python para:
        \begin{itemize}
            \item Extração de quadros de vídeos com FFmpeg.
            \item Conversão de imagens para escala de cinza com OpenCV.
            \item Manipulação de imagens com PIL.
        \end{itemize}
    \end{itemize}
\end{frame}

% Slide: Extração de Quadros Usando FFmpeg e Subprocess
\begin{frame}[fragile]{Extração de Quadros Usando FFmpeg e Subprocess}
    \begin{itemize}
        \item O FFmpeg é uma ferramenta de linha de comando para manipulação de vídeos, que pode ser chamada do Python com \texttt{subprocess}.
        \item Permite extrair quadros de vídeos para processamento posterior.
    \end{itemize}
    
\begin{lstlisting}
import subprocess

video_path = 'video.mp4'
output_pattern = 'frame_%04d.png'

# Comando FFmpeg para extrair quadros do video
command = [
    'ffmpeg', '-i', video_path,
    '-vf', 'format=rgb24',
    output_pattern
]

# Executa o comando usando subprocess
subprocess.run(command, check=True)
\end{lstlisting}

    \begin{itemize}
        \item O comando salva os quadros como \texttt{frame\_0001.png}, \texttt{frame\_0002.png}, etc.
    \end{itemize}
\end{frame}

% Slide: Conversão para Escala de Cinza com OpenCV
\begin{frame}[fragile]{Conversão para Escala de Cinza com OpenCV}
    \begin{itemize}
        \item Use o OpenCV para converter as imagens RGB extraídas para escala de cinza.
    \end{itemize}
    
\begin{lstlisting}
import cv2
import glob

image_files = glob.glob('frame_*.png')

for image_file in image_files:
    img = cv2.imread(image_file)
    gray_img = cv2.cvtColor(img, cv2.COLOR_BGR2GRAY)
    gray_filename = 'gray_' + image_file
    cv2.imwrite(gray_filename, gray_img)
\end{lstlisting}

    \begin{itemize}
        \item O código lê cada imagem RGB e converte para escala de cinza.
    \end{itemize}
\end{frame}

% Slide: Manipulação de Imagens com PIL
\begin{frame}[fragile]{Manipulação de Imagens com PIL}
    \begin{itemize}
        \item O PIL (Python Imaging Library) permite abrir, manipular e salvar imagens.
    \end{itemize}
    
\begin{lstlisting}
from PIL import Image
import glob

image_files = glob.glob('frame_*.png')

for image_file in image_files:
    img = Image.open(image_file)
    gray_img = img.convert('L') # 'L' para escala de cinza
    gray_filename = 'gray_' + image_file
    gray_img.save(gray_filename)
\end{lstlisting}

    \begin{itemize}
        \item Este exemplo converte as imagens para escala de cinza e as salva com um prefixo \texttt{gray\_}.
    \end{itemize}
\end{frame}

% Slide: Submissão da Tarefa Usando Git
\begin{frame}[fragile]{Submissão da Tarefa Usando Git}
    \begin{itemize}
        \item Para enviar a tarefa, siga os passos abaixo:
        \begin{enumerate}
            \item Crie uma nova branch: \texttt{git checkout -b sua-branch}
            \item Adicione os arquivos: \texttt{git add .}
            \item Faça o commit: \texttt{git commit -m "Tarefa de proc. de imagens"}
            \item Envie para o remoto: \texttt{git push origin sua-branch}
            \item Crie um Pull Request (PR) no GitHub para a branch principal.
        \end{enumerate}
        \item Certifique-se de que o código esteja bem comentado e organizado.
    \end{itemize}
\end{frame}

% Slide: Conclusão e Próximos Passos
\begin{frame}[standout]
    Conclusão e Próximos Passos
\end{frame}

\begin{frame}{Próximos Passos}
    \begin{itemize}
        \item Aprendemos a usar o FFmpeg com \texttt{subprocess}, além de técnicas de manipulação de imagens com OpenCV e PIL.
        \item Nos próximos encontros, exploraremos técnicas avançadas de análise de vídeos e tracking de objetos.
    \end{itemize}
\end{frame}

\end{document}