\documentclass[aspectratio=169]{beamer}

% Pacotes básicos e idioma
\usepackage[utf8]{inputenc}
\usepackage[portuguese]{babel}

% --- CHAMA O PACOTE DE ESTILO CUSTOMIZADO ---
\usepackage{preto}

% Caminho para as imagens 
\graphicspath{{./images/}}

% --- Informações da Capa ---
\title{Visão Computacional e Deep Learning na Análise do Movimento de Atletas}
\subtitle{Perspectivas para a análise de desempenho}
\date{IV Simpósio de Tecnologia Aplicada à Análise de Desempenho Esportivo \\ 27 e 28 de Fevereiro de 2026}
\author{Prof. Dr. Paulo Roberto Pereia Santiago}
\institute{Escola de Educação Física e Esporte de Ribeirão Preto (EEFERP) - USP \\ Laboratório de Biomecânica e Controle Motor (LaBioCoM)}

\begin{document}

\begin{frame}
    \titlepage
\end{frame}

\begin{frame}{Agenda}
    \tableofcontents
\end{frame}

% ==========================================
% BLOCO 1: O Contexto, a Dor do Esporte e a Soberania Tecnológica
% ==========================================
\section{O Contexto e a Dor do Esporte}

\begin{frame}{Soberania Tecnológica: Criadores vs. Consumidores}
    \begin{itemize}
        \item O mercado esportivo brasileiro é historicamente dependente de tecnologias importadas (EUA, Europa, Austrália).
        \item Consumo de hardwares caríssimos e softwares fechados (\textit{Black Boxes}).
        \item O atual boom da Inteligência Artificial: ChatGPT, Gemini, Claude, Grok... onde está a tecnologia nacional?
        \item A urgência de deixarmos de ser apenas ``usuários \textit{premium}'' para nos tornarmos \textbf{desenvolvedores} das nossas próprias soluções.
    \end{itemize}
\end{frame}


% --- Slide 1: A Crítica ao Consumo Passivo ---
\begin{frame}{A Ilusão do Domínio: Somos Apertadores de Botões?}
    \begin{columns}[T]
        \begin{column}{0.65\textwidth}
            \begin{itemize}
                \item Temos a facilidade de comprar o \textit{software} mais caro e o \textit{hardware} mais moderno.
                \item \textbf{O Problema:} Sabemos onde clicar, mas não sabemos o que acontece no ``silício''.
                \item \textit{Analogia da Memória:} Tratamos a tecnologia como uma caixa preta de \textit{bits} consumíveis, não como geração de conhecimento.
            \end{itemize}
        \end{column}
        
        \begin{column}{0.35\textwidth}
            \centering
            \includegraphics[height=5.5cm, keepaspectratio]{pauloef_perdido.png}
        \end{column}
    \end{columns}
\end{frame}

% --- Slide 2: A Camada de Desenvolvimento (O Motor) ---
\begin{frame}{A Ilusão do Domínio: O que há por trás do botão?}
    \begin{center}
        \includegraphics[height=6.5cm, keepaspectratio]{paper1.png}
    \end{center}
\end{frame}

% --- Slide 3: O Nível do Detalhe (A Provocação) ---
\begin{frame}{A Ilusão do Domínio: A Ciência do Pixel}
    \begin{center}
        \includegraphics[height=4.5cm, keepaspectratio]{onepixel.png}
    \end{center}
    
    \vfill

    \begin{center}
        \begin{tcolorbox}[
            colback=red!5,
            colframe=red!75!black,
            title=Provocação Metodológica,
            width=0.95\textwidth,
            top=2pt, bottom=2pt
        ]
            \centering 
            \small \textbf{Como vamos discutir a complexidade da análise de desempenho se não sabemos como funciona o motor do algoritmo?} 
        \end{tcolorbox}
    \end{center}
\end{frame}

% --- Slide 4: A Matriz de Dados (A Realidade) ---
\begin{frame}{A Ilusão do Domínio: A Realidade dos Dados Brutos}
    \begin{center}
        \includegraphics[height=7cm, keepaspectratio]{images/terminal_xxd.png}
    \end{center}
\end{frame}

% --- Slide 5: O Hype / O Falso Domínio ---
\begin{frame}{Separando o \textit{Hype} da Ciência: O Falso Domínio}
    \begin{center}
        \includegraphics[height=4.2cm, keepaspectratio]{neobullshit.png}
        
        \vspace{0.2cm}
        \textbf{O \textit{Hype} Tecnológico} \\
        \small O perigo das ``caixas pretas'' e promessas mágicas sem validação.
    \end{center}
\end{frame}

% --- Slide 6: A Ciência / O Desafio Real ---
\begin{frame}{Separando o \textit{Hype} da Ciência: O Pálido Ponto Azul}
    \begin{center}
        \includegraphics[height=4.2cm, keepaspectratio]{Pale_Blue_Dot.png}
        
        \vspace{0.2cm}
        \textbf{O Pálido Ponto Azul} \\
        \small O rigor metodológico: encontrar e rastrear o sinal (\textit{pixel}) verdadeiro no meio do ruído cósmico (ou da quadra).
    \end{center}
\end{frame}

\begin{frame}{A Nossa Resposta: Desenvolvimento e Validação Nacional}
    \begin{itemize}
        \item Rompendo a dependência de "caixas pretas" por meio da construção e validação de algoritmos de código aberto (\textit{open-source}).
        \item \textbf{Futebol de Campo:} Validação de métodos \textit{markerless} baseados em \textit{machine learning} contra a digitalização 3D tradicional para a análise cinemática do chute \cite{vieira2022automatic}.
        \item \textbf{Desportos de Combate:} Avaliação da precisão de arquiteturas como o \textit{OpenPose} para rastreamento de atletas em áreas oficiais de Taekwondo \cite{banks2024accuracy}.
        \item \textbf{Movimentos Complexos:} Desenvolvimento de \textit{pipelines} próprios (baseados em \textit{MediaPipe}) para quantificação da cinemática 2D em exercícios como o agachamento \cite{pereira2025markerless}.
    \end{itemize}
\end{frame}

% --- Slide Novo: Capa do Artigo Vieira et al. (2022) ---
\begin{frame}{Nossa Publicação: \textit{Markerless} no Futebol}
    \begin{center}
        \includegraphics[height=6.5cm, keepaspectratio]{vieira_capa.png}
    \end{center}
\end{frame}


% --- Slide Novo: Imagens do Artigo Vieira et al. (2022) ---
\begin{frame}{Validação em Cenário Real: O Chute no Futebol}
    \begin{itemize}
        \item Comparação visual do método \textit{markerless} automático contra a digitalização manual 3D tradicional \cite{vieira2022automatic}.
    \end{itemize}
    
    \vspace{0.3cm}
    
    \begin{columns}[c] % Alinhamento centralizado das colunas
        
        % Coluna Esquerda: Primeira Figura
        \begin{column}{0.5\textwidth}
            \begin{center}
                \includegraphics[width=0.95\textwidth, keepaspectratio]{vieira_imag1.png}
            \end{center}
        \end{column}
        
        % Coluna Direita: Segunda Figura
        \begin{column}{0.5\textwidth}
            \begin{center}
                \includegraphics[width=0.95\textwidth, keepaspectratio]{vieira_imag2.png}
            \end{center}
        \end{column}
        
    \end{columns}
\end{frame}


\begin{frame}{Código Aberto: Entendendo o Motor na Prática}
    \begin{itemize}
        \item Para romper a cultura da "caixa preta", todo o código discutido aqui é \textit{open-source}.
        \item Desenvolvemos um \textit{pipeline} completo em \textit{Python} (utilizando \textit{PyTorch}) para provar o conceito básico de \textit{Deep Learning}.
    \end{itemize}
    
    \vspace{0.3cm}
    
    \textbf{O que você encontrará no repositório (\texttt{src/}):}
    \begin{itemize}
        \item \textbf{Demonstração didática:} \texttt{img7x7.py} (matriz no terminal e bits); \texttt{png2raw.py} (PNG $\rightarrow$ RAW).
        \item \textbf{Geração de Dados:} Construção de matrizes e \textit{datasets} sintéticos (\texttt{makedataset.py}).
        \item \textbf{O Treinamento:} Uma Rede Neural Convolucional (CNN) feita do zero para prever coordenadas espaciais (\texttt{trainblack7x7.py}).
        \item \textbf{A Inferência:} Ferramentas de terminal (CLI) para testar o rastreamento em novas imagens (\texttt{predictblackdot.py}).
    \end{itemize}
    
    \vspace{0.4cm}
    
    \begin{center}
        \begin{tcolorbox}[
            colback=blue!5,
            colframe=blue!75!black,
            title=Acesse, Clone e Modifique:,
            boxsep=2pt, top=4pt, bottom=4pt, width=0.98\textwidth
        ]
            \centering
            \small \url{https://github.com/paulopreto/CompVison_ML_MovSport_2026_USP/tree/main/src}
        \end{tcolorbox}
    \end{center}
\end{frame}

\begin{frame}{O Paradigma Tradicional vs. Validade Ecológica}
    \begin{columns}
        \begin{column}{0.5\textwidth}
            \textbf{Padrão-Ouro (Laboratório)}
            \begin{itemize}
                \item Captura optoeletrônica com marcadores.
                \item Alta precisão, mas restrito a um ambiente rigidamente controlado.
                \item Invasivo e inviável em situação real de jogo.
            \end{itemize}
        \end{column}
        \begin{column}{0.5\textwidth}
            \textbf{Demanda Ecológica (Campo)}
            \begin{itemize}
                \item Avaliação durante o treino ou competição real.
                \item Necessidade de respostas rápidas para a equipe técnica.
                \item \textit{A "dor" do clube}: horas e horas de vídeo armazenadas, mas pouca capacidade de extrair métricas biomecânicas acionáveis.
            \end{itemize}
        \end{column}
    \end{columns}
\end{frame}

\begin{frame}{O Paradigma Tradicional vs. Validade Ecológica}
    \begin{columns}
        \begin{column}{0.5\textwidth}
            \textbf{Padrão-Ouro (Laboratório)}
            \begin{itemize}
                \item Captura com marcadores (ex: Vicon).
                \item Alta precisão, mas restrito ao ambiente controlado.
                \item Invasivo e inviável em situação de jogo.
            \end{itemize}
        \end{column}
        \begin{column}{0.5\textwidth}
            \textbf{Demanda Ecológica (Campo)}
            \begin{itemize}
                \item Avaliação durante o treino ou competição.
                \item Necessidade de respostas rápidas para a comissão técnica.
                \item \textit{A "dor" do clube}: ter horas de vídeo, mas poucos dados acionáveis.
            \end{itemize}
        \end{column}
    \end{columns}
\end{frame}


% ==========================================
% BLOCO 2: O Arsenal Tecnológico
% ==========================================
\section{O Arsenal Tecnológico: Visão Computacional e Deep Learning}

\begin{frame}{A Revolução da Captura sem Marcadores}
    \begin{itemize}
        \item Como as Redes Neurais Profundas viabilizaram a \textit{Markerless Motion Capture}.
        \item Democratização da extração de variáveis cinemáticas a partir de câmeras RGB comuns.
        \item Superando os ruídos e calibrando a predição para movimentos esportivos complexos.
    \end{itemize}
\end{frame}

\begin{frame}{Detecção e Rastreamento com YOLO}
    \begin{itemize}
        \item \textbf{YOLO (You Only Look Once):} Eficiência para identificar e rastrear múltiplos atletas em tempo real.
        \item O desafio da oclusão constante em desportos coletivos.
        \item Associações temporais e espaciais para manter a identidade do atleta ao longo do vídeo.
    \end{itemize}
    % Sugestão: Inserir um GIF/Vídeo curto do YOLO plotando as bounding boxes nos atletas
    \begin{center}
        \textit{[Inserir Vídeo/GIF do YOLO rastreando atletas aqui]}
    \end{center}
\end{frame}

\begin{frame}{Estimativa de Pose com MediaPipe}
    \begin{itemize}
        \item Modelagem do esqueleto 2D e 3D diretamente da imagem.
        \item Cálculo de ângulos articulares, velocidades e assimetrias na marcha/corrida.
        \item Da imagem bruta para o modelo biomecânico: validando os pontos anatômicos.
    \end{itemize}
    % Sugestão: Inserir um GIF do MediaPipe
    \begin{center}
        \textit{[Inserir Vídeo/GIF do MediaPipe extraindo o esqueleto de um atleta]}
    \end{center}
\end{frame}


% ==========================================
% BLOCO 3: Mão na Massa, Projetos Nacionais e Ferramentas Abertas
% ==========================================
\section{Mão na Massa: Validações Científicas do LaBioCoM}

\begin{frame}{Rastreamento \textit{Markerless} em Cenários Reais}
    \begin{itemize}
        \item \textbf{Desportos de Combate:} Validação do \textit{OpenPose} para estimar a posição e movimentação de atletas de Taekwondo em área oficial de combate \cite{banks2024accuracy}.
        \item \textbf{Futebol de Campo:} Método automático e não invasivo de \textit{machine learning} (baseado em \textit{OpenPose}) validado contra a digitalização manual 3D para a análise cinemática do chute \cite{vieira2022automatic}.
        \item A transição bem-sucedida dos laboratórios fechados para o ambiente natural (\textit{in the wild}).
    \end{itemize}
\end{frame}

\begin{frame}{Validação Clínica e Acesso Aberto}
    \begin{itemize}
        \item \textbf{Cinemática 2D:} Desenvolvimento e validação de um \textit{pipeline} baseado em pixels, rodando o \textit{MediaPipe}, para quantificar movimentos de membros inferiores no agachamento \cite{pereira2025markerless}.
        \item Uma alternativa consistente e de baixo custo para auxiliar a tomada de decisão clínica e desportiva onde a captura de movimento tradicional não está disponível.
        \item A força da ciência aberta (\textit{open-source}) para reduzir a desigualdade tecnológica e fomentar o desenvolvimento nacional.
    \end{itemize}
\end{frame}

\begin{frame}[fragile]{O \textit{vailá} Multimodal Toolbox \cite{santiago2024vaila}}
    \begin{itemize}
        \item Um ecossistema em \textit{Python} desenvolvido para solucionar a complexidade da integração de dados biomecânicos.
        \item O \textit{vailá} liberta o pesquisador e o analista de desempenho das "caixas pretas" do mercado.
    \end{itemize}
    

\end{frame}

\begin{frame}[fragile]{O \textit{vailá} Multimodal Toolbox}
    \begin{itemize}
        \item Desenvolvido para resolver o "pesadelo" da integração de dados cinemáticos e dinâmicos de diferentes fontes.
        \item Uma solução prática, em Python, para organizar pipelines de pesquisa biomecânica.
    \end{itemize}
    
\href{https://github.com/vaila-multimodaltoolbox/vaila}{vailá}
\end{frame}


% ==========================================
% BLOCO 4: Perspectivas e Futuro
% ==========================================
\section{A Fronteira do Conhecimento e Perspectivas}

\begin{frame}{A Provocação do VAR Semiautomático}
    \begin{itemize}
        \item A \textit{Premier League} adotou dezenas de \textit{smartphones} comuns para a Tecnologia de Impedimento Semiautomático (SAOT).
        \item \textbf{A Provocação:} Se a liga mais rica do mundo extrai métricas de elite usando celulares, o segredo não é o \textit{hardware} de 100 mil dólares, é o \textbf{algoritmo}.
        \item \textit{Se é possível fazer com um celular de prateleira, nós também podemos fazer aqui!}
    \end{itemize}
    
    \vspace{0.4cm}
    \textbf{Veja na prática:} \\
    \scriptsize
    \url{https://www.youtube.com/watch?v=af6P0P25HQk} \\
    \url{https://www.youtube.com/shorts/kxUme_SgOzc}
    
    \vspace{0.4cm}
    \begin{center}
        \begin{tcolorbox}[colback=blue!5,colframe=blue!75!black,boxsep=2pt, top=2pt, bottom=2pt, width=0.95\textwidth]
            \centering \small O diferencial competitivo migrou da câmera para a inteligência em \textit{software}.
        \end{tcolorbox}
    \end{center}
\end{frame}

\begin{frame}{Fusão Multimodal e Dispositivos IoT}
    \begin{itemize}
        \item A Visão Computacional não trabalha sozinha.
        \item Sincronização de dados de vídeo com \textit{wearables}, sensores inerciais e dispositivos instrumentados (ex: blocos de partida IoT).
        \item O panorama completo do esforço e desempenho do atleta no cenário real.
    \end{itemize}
\end{frame}

\begin{frame}{Identificação de Talentos e Simulações Computacionais}
    \begin{itemize}
        \item Aplicação de \textit{machine learning} na identificação segura de talentos esportivos.
        \item O conceito de \textit{safe deselection}: cruzando evidências empíricas e simulações para evitar perdas de atletas com potencial de maturação tardia.
        \item A tecnologia reduzindo vieses humanos no processo de seleção.
    \end{itemize}
\end{frame}

\begin{frame}
    \begin{center}
        \begin{tcolorbox}[
            colback=green!5,
            colframe=green!50!black,
            title=\textit{Take-home Message},
            width=0.95\textwidth,
            boxsep=5pt, top=8pt, bottom=8pt
        ]
            \centering \Large 
            A inteligência artificial e a visão computacional não substituem o treinador; elas fornecem a lente de precisão para a sua intuição e experiência.
        \end{tcolorbox}
    \end{center}
\end{frame}
\begin{frame}[standout]
    Obrigado! \\
    \vspace{1em}
    \normalsize \textbf{Prof. Dr. Paulo Santiago} \\
    \normalsize EEFERP-USP / LaBioCoM \\
    \href{https://github.com/paulopreto/CompVison_ML_MovSport_2026_USP}{}
\end{frame}

% --- Slide de Referências ---
\begin{frame}[allowframebreaks]{Referências}
    \bibliographystyle{apalike} % Estilo da citação (ex: apalike, IEEEtran, etc)
    \bibliography{references}   % Chama o seu arquivo references.bib
\end{frame}

\end{document}